% ----------------------------------------------------------
\chapter{Entregas}
% ----------------------------------------------------------

O desenvolvimento do Projeto Integrador de Sistemas Inteligentes II resultará em um conjunto de entregáveis que materializam todo o processo de concepção, implementação e validação da solução proposta. Estes entregáveis são fundamentais para a avaliação da disciplina, para a consolidação do aprendizado dos discentes e para a prestação de contas junto às organizações parceiras e demais stakeholders envolvidos. De forma geral, serão produzidos três tipos principais de entregas: (1) um projeto envolvendo ciência de dados, desenvolvido para atender a uma solução proposta; (2) uma descrição formal do projeto, contendo toda a documentação técnica e científica; (3) o Diário de Bordo, apresentando os passos do desenvolvimento do projeto; e (4) uma apresentação detalhada, que sintetiza e comunica os resultados obtidos, bem como as técnicas e algoritmos utilizados. Cada uma dessas entregas é descrita a seguir com maior detalhamento.

\section{Projeto envolvendo ciência de dados}

O principal produto do trabalho será o desenvolvimento de um artefato tecnológico aplicado, construído a partir de técnicas de ciência de dados, análise de sistemas inteligentes e metodologias de tratamento de informação. Esse projeto deverá responder de maneira objetiva a um problema real previamente identificado em uma organização parceira ou estar alinhado a um dos Objetivos de Desenvolvimento Sustentável (ODS) da ONU.

O artefato poderá assumir diferentes formatos, dependendo da natureza do desafio enfrentado: desde um sistema de coleta e análise de dados (por exemplo, um pipeline para ingestão e limpeza de dados), passando pela construção de modelos de aprendizado de máquina (ex.: classificação, regressão, agrupamento ou recomendação), até a implementação de um painel interativo de visualização que auxilie gestores na tomada de decisão.

\section{Diário de bordo}

O diário de bordo será um registro detalhado de cada passo percorrido durante o processo de desenvolvimento do projeto, elaborado de forma contínua ao longo do período letivo. Ele deverá apresentar, semana a semana, os avanços alcançados, as dificuldades encontradas e as soluções propostas pelo grupo. Esse documento não é apenas um relatório, mas um recurso reflexivo e de acompanhamento, permitindo identificar como as decisões foram tomadas e como o trabalho evoluiu com o tempo.

\section{Descrição formal do projeto}

Será produzido um documento técnico contendo a descrição formal do projeto. Esse documento deverá apresentar de forma clara e objetiva todas as fases do trabalho, incluindo: objetivo principal, hipóteses levantadas, justificativa da escolha do tema, revisão teórica, metodologia adotada e resultados esperados. A descrição formal funciona como um guia escrito, permitindo que qualquer avaliador compreenda a lógica da pesquisa e suas fundamentações.

\section{Apresentação detalhada}

Será preparada uma apresentação final que sintetize de forma didática todas as etapas do projeto, as técnicas empregadas e os algoritmos utilizados. Esse material deverá ser produzido com recursos visuais (como slides, gráficos, fluxogramas e tabelas), destacando as principais descobertas, os métodos aplicados e os resultados alcançados. Além disso, a apresentação deverá estar estruturada de modo a comunicar o conteúdo para públicos diferentes, como professores, colegas e possíveis avaliadores externos.