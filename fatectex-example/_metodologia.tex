% ----------------------------------------------------------
\chapter{Metodologia}
\label{cap:metodologia}
% ----------------------------------------------------------

O projeto será desenvolvido em grupos de trabalho e dividido nas seguintes etapas:

Definição do problema: 

\begin{enumerate}
    \item Identificação de tema e problema vinculado aos ODS da ONU ou desafios propostos por organizações parceiras, com suporte do professor orientador.
    \item Levantamento de requisitos: Análise detalhada do cenário do desafio, identificando as necessidades da organização.
    \item Modelagem da solução: Desenvolvimento de arquitetura para armazenamento e processamento de dados.
    \item Implementação de algoritmos: Criação de algoritmos de conexão, análise e processamento de dados.
    \item Avaliação dos resultados: Testes e validação da solução desenvolvida frente ao problema identificado.
    \item Documentação: Elaboração de relatórios técnicos e artigos científicos detalhando todas as etapas e resultados.
    \item Apresentação: Comunicação dos resultados para as partes interessadas do projeto, incluindo a organização parceira.
\end{enumerate}


\section{Ferramentas e técnicas utilizadas}

\begin{itemize}
    \item Figma, Adobe XD, Sketch ou outra ferramenta de design de interfaces digitais (UI/UX).
    \item Python, R ou outra linguagem para ciência de dados.
    \item Plataformas de armazenamento e processamento de dados (ex.: SQL, NoSQL, Hadoop).
    \item Técnicas de inteligência artificial e machine learning conforme o problema definido.
    \item Ferramentas de documentação e apresentação científica (Word, LaTeX, PowerPoint).
    \item Ferramentas de versionamento de código/projeto Git.
    \item Ferramentas de gestão Kanban/Scrum; reuniões semanais; Diário de Bordo.
    \item LGPD/Ética, base legal, minimização de dados, anonimização, controle de acesso.
\end{itemize}

\section{Componentes Curriculares Envolvidos}

O desenvolvimento do projeto integrador em ciência de dados exige a mobilização de diferentes componentes curriculares, de modo a proporcionar uma formação interdisciplinar e aplicada. Cada componente contribui de forma específica para a construção do conhecimento e para a efetiva resolução do problema proposto. A seguir, detalha-se a participação dos principais componentes curriculares envolvidos:

    \subsection{Estatística e Probabilidade}
    
        Este componente fornece a base necessária para a análise de dados, permitindo que os estudantes compreendam conceitos como média, mediana, variância, desvio padrão, distribuições de probabilidade e testes de hipóteses. No projeto, os alunos aplicarão técnicas estatísticas para explorar os dados e identificar padrões. Por exemplo, ao analisar o consumo de energia em uma empresa, será possível calcular médias semanais, identificar valores atípicos e verificar se há sazonalidade nas informações.
    
    \subsection{Programação e Estruturas de Dados}
    
        A programação é um eixo central do projeto, pois permite a manipulação e transformação dos dados. O conhecimento em estruturas de dados — listas, filas, pilhas, árvores e grafos — possibilita a criação de algoritmos eficientes para a solução proposta. Em um cenário prático, os alunos poderão utilizar linguagens como Python para implementar algoritmos de classificação (como decision trees) ou de regressão, organizando os dados de forma a facilitar o processamento e a análise.

    \subsection{Banco de Dados}
        Os conceitos de modelagem, armazenamento e recuperação de dados são fundamentais para o trabalho com grandes volumes de informação. Os alunos aprenderão a construir consultas SQL, normalizar tabelas e integrar diferentes fontes de dados. Por exemplo, em um projeto voltado à análise de vendas no comércio eletrônico, será necessário acessar registros de compras, clientes e produtos armazenados em bancos de dados relacionais, garantindo consistência e integridade das informações.

    \subsection{Mineração de Dados e Aprendizado de Máquina}
        Este componente curricular possibilita a utilização de algoritmos de classificação, agrupamento, regressão e associação para a descoberta de conhecimento nos dados. Os estudantes aprenderão a selecionar técnicas adequadas para cada problema. Como exemplo, em um projeto de previsão de evasão escolar, pode-se empregar algoritmos como Random Forest ou K-Means para identificar grupos de alunos em situação de risco.

    \subsection{Comunicação e Expressão}
        Um projeto de ciência de dados só se torna efetivo quando os resultados são comunicados de forma clara e acessível. Este componente curricular contribui para o desenvolvimento de relatórios técnicos, apresentações orais e visuais, além da elaboração de textos acadêmicos. No projeto, os estudantes deverão transformar resultados técnicos em linguagem compreensível para gestores, por meio de gráficos, infográficos e resumos executivos.

    \subsection{Ética e Legislação em Tecnologia}
        O uso de dados envolve aspectos éticos e legais que não podem ser negligenciados. Os alunos discutirão questões de privacidade, segurança e uso responsável da informação, considerando legislações como a Lei Geral de Proteção de Dados (LGPD). Em um exemplo prático, ao trabalhar com dados de saúde de pacientes, será necessário anonimizar as informações e obter consentimento adequado, respeitando a confidencialidade.

    \subsection{Metodologia Científica}
        A organização do projeto segue princípios da pesquisa científica, com definição clara de objetivos, hipóteses, métodos e resultados esperados. Esse componente curricular fornece a base para a elaboração do relatório final e para a condução das etapas do diário de bordo. Por exemplo, os alunos deverão registrar semanalmente as atividades realizadas, justificando metodologicamente as escolhas feitas \cite{abntex2cite}.

    \subsection{Integração dos Componentes}

        A interdisciplinaridade é a essência do projeto. Um problema real, como a previsão de demanda de produtos em um supermercado, requer a utilização conjunta de conceitos de estatística (para análise descritiva), programação (para implementação de modelos), banco de dados (para acesso às informações), mineração de dados (para previsão), comunicação (para a apresentação dos resultados), ética (para tratamento correto dos dados de clientes) e metodologia científica (para estruturação e documentação do processo). Dessa forma, cada componente curricular complementa o outro, formando um ciclo completo de aprendizagem prática.




