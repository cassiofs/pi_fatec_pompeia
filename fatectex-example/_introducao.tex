% ----------------------------------------------------------
\chapter{Introdução}
\label{cap:intr}
% ----------------------------------------------------------

Desenvolver um trabalho prático baseado em problema que integre as teorias abordadas nas disciplinas do 1º semestre. Este trabalho precisa ser baseado em uma necessidade tecnológica de uma organização real, ou deve colaborar com um dos Objetivos de Desenvolvimento Sustentáveis (ODS) da Organização das Nações Unidas (ONU) voltando-se no auxílio direto para a sociedade.

O trabalho a ser desenvolvido consiste em desenvolver projetos para coleta, gerenciamento e processamento de dados por meio de algoritmos computacionais.


% ----------------------------------------------------------
\chapter{Caracterização do Projeto}
\label{cap:caracterizacao}
% ----------------------------------------------------------

Nesta seção são apresentados os elementos fundamentais que compõem a estrutura do projeto de extensão, fornecendo uma visão inicial sobre sua organização, abrangência e público de interesse. O objetivo é contextualizar a proposta, destacando os aspectos centrais que orientam sua execução e justificam sua relevância para a comunidade acadêmica e externa.

São descritos a seguir a carga horária prevista, o público-alvo, o parceiro e stakeholders envolvidos, bem como o diagnóstico inicial que fundamenta a necessidade e pertinência da iniciativa. Essas informações permitem compreender a dimensão do projeto e estabelecer a base para o desenvolvimento das atividades planejadas.

\section{Carga horária}

Um total de 80 horas/aula será aplicado na disciplina Projeto Integrador II, integrado com os demais componentes curriculares do 2º semestre.

\section{Público-alvo}

Organizações, empresas, associações e organizações não governamentais parceiras da Fatec.

\section{Parceiro, Stakeholders e Diagnóstico Inicial}

\begin{itemize}
    \item Organização parceira: missão, processos relevantes, dados disponíveis.
    \item Stakeholders: quem usa, quem decide, quem mantém (matriz RACI opcional).
    \item Restrições: orçamento, prazos, infraestrutura, LGPD/ética.
\end{itemize}

Exemplo:

Parceiro: Cooperativa Verde (80 colaboradores)...

Stakeholders: triadores (U), gerente (A), TI municipal (C), equipe do projeto (R)...

Restrições: sem internet no galpão; dados sensíveis (biometria NÃO coletada)...

